\documentclass[a4paper,10pt]{article}
\usepackage{graphicx,hyperref}
\usepackage[top=90pt,bottom=90pt,left=90pt,right=100pt]{geometry}

\begin{document} 

\clearpage
\thispagestyle{empty}

\begin{figure}
\hspace{10cm} \includegraphics[clip = true, viewport = 0cm 0cm 4.5cm 3.5cm, scale = 1]{SIAMSC_Delft} 
\end{figure}

\textbf{Funding Request for Student Krylov Day 2015}
\bigskip
\bigskip

Dear Prof. Van Horssen, \hfill \today

\bigskip
We have recently founded a \textit{SIAM Student Chapter} at our university. Our Student Chapter is the first division of SIAM in The Netherlands, and it is meant to bring together MSc and PhD students, postdocs and professors from all mathematically-oriented departments. The Chapter organizes seminars, lectures (both on research and on software tools), company visits and social events. Our faculty advisors are Martin van Gijzen and Kees Vuik and we currently have about 45 student members, mostly from the Numerical Analysis section.

\bigskip
\noindent Currently involved departments at TU Delft are:
\begin{itemize}
 \item Applied Mathematics (sections Statistics, Mathematical Physics, Numerical Analysis),
 \item Aerospace Engineering (section Aerodynamics),
 \item Electrical Engineering (section Circuits \& Systems and Intelligent Electrical Power Grids),
 \item Marine and Transport Technology (section Offshore \& Dredging Engineering),
 \item Geoscience \& Engineering (section Petroleum engineering),
\end{itemize}
and we hope to attract further departments in the future.

\bigskip
SIAM supports our Student Chapter with 500 dollars per year. This money is used to promote the Student Chapter (posters, homepage), and to pay food and drinks during our meetings. Furthermore, external speakers will be invited, whose travel we need to pay. In addition to our monthly meetings, we are organizing a \textbf{one-day workshop on Krylov methods} which will be held on February 2, 2015 at TU Delft. Therefore, we invited PhD students from different European countries who will present their recent work on Krylov methods. Moreover, we would like to organize a joint diner in Delft at the end of the workshop and offer refreshments during the day. More information can be found at our homepage:
\begin{center}
 \url{http://sscdelft.github.io/activities/2015/02/02/krylov-day.html}
\end{center}

% \bigskip
% Many succesfull Student Chapters also obtain yearly funding from their own university. In this way, the Student Chapters can reach out to more different departments and invite more people to their talks. Is it possible that our Chapter also obtains some regular funding from the TU Delft?

% \bigskip
% If you have any further enquiries, please do not hesitate to contact us.

\bigskip
\noindent For the realization of this workshop, we kindly ask you for additional financial support.

\bigskip
\noindent Yours Sincerely,

\bigskip 
\bigskip

\noindent Manuel Baumann (President)

\noindent Reinaldo Astidullo (Vice-President)

\noindent Thea Vuik (Secretary and Treasurer)

\noindent \url{http://sscdelft.github.io}

\newpage
Please find attached our estimate of costs:
\begin{table}[h]
\centering
\begin{tabular}{lccr}
 Activity & Cost estimate & Participants & Total Costs  \\
 \hline
 Lunch at Aula & 10 EUR  & 20  & 200 EUR   \\
 Refreshments & 5 EUR & 20 & 120 EUR \\
 Diner in city center & 20 EUR & 10  & 200 EUR   \\
 Book of abstracts (10pages)& 1 EUR  & 25 & 25 EUR \\
 & & & \textbf{545 EUR}
\end{tabular}
\end{table}


\end{document}